\documentclass{article}
\usepackage{booktabs}
\usepackage{url}
\usepackage{hyperref}
\usepackage{fancyhdr}

\pagestyle{fancy}

\begin{document}
\pagenumbering{gobble}
\lhead{Team 6\\ JSTanks\\}
\newpage
\title{JSTanks - Development Plan}
\date{December 8, 2016}
\author{Jiahao Li\\LI577\\001416646\and Pavithran Pathmarajah\\PATHMAP\\
001410729 \and Viren Patel\\PATELVH3\\001419057}

\maketitle

\newpage
\pagenumbering{arabic}

\section*{Revision History}
\begin{table}[h!]
  \centering
  \caption{Revision History}
  \label{tab:table1}
  \begin{tabular}{clcc}
	\toprule
	Date &  Developer & Change & Revision\\
	\midrule
	September 29&Jiahao Li &Initial Draft &0\\
	September 29&Pavithran Pathmarajah &Initial Draft &0\\
	September 29&Viren Patel  &Initial Draft &0\\
	\midrule
	December 4&Pavithran Pathmarajah &Update Plans &1\\
	\bottomrule
  \end{tabular}
\end{table}



\newpage
\pagenumbering{arabic}

\section*{SE 3XA3: Development Plan}

\begin{table}[h!]

  \begin{tabular}{r l}
	\toprule
	Name &  MacID \\
	\midrule
	Jiahao Li & li577 \\
	Pavithran Pathmarajah & pathmap\\
	Viren Patel  & patelvh3\\
	\bottomrule
  \end{tabular}
\end{table}


\subsection*{Meeting Plan:}  
The team has decided to meet twice a week outside of laboratory hours, this
means that the team will be meeting 4 times each week. The labs which take place
on Monday and Thursday in ITB 236 will be our primary meeting?s for the week,
where we will be able to update the TA on our group status, and deal with major
issues. Then we will have two progress meetings each week, every Wednesday we
book a room in Thode Library to hold progress meetings at 5:30pm; during these
meetings we will monitor the groups weekly progress in order to gauge our
ability to meet our milestones on time. Our final meeting will take place on
Sunday evenings virtually through Facebook, where we will delegate weekly tasks
for the group members for that week. The team will rotate meeting facilitators,
but for the Sunday meetings Pavithran Pathmarajah will be the designated
delegator for the duration of the project.

\subsection*{Team Communication Plan:}  
The primary means of communication our
team will be using throughout this project is a Facebook group chat. The
Facebook group will be used for sharing ideas, research, questions and issues
that are discovered while working individually. However, the implementation
(including programming issues) and documentation will be communicated through
GitLab. We are also using google docs as a means to work together on the
documents involved in this project.

\subsection*{Team Member roles:}
\begin{table}[h!]     
	\begin{tabular}{r l}    
		Pavithran Pathmarajah: & Team Lead\\   
		Jiahao Li: & Scribe\\   
		Viren Patel: & JS-Specialist
	\end{tabular}
\end{table}

	
	
\subsection*{Git workflow plan:}  
A good Git Workflow is key to the success of a
project and must be determined before all else; our team has decided to go with
a simple Git workflow where in we branch off of the master. Using this branch as
a central branch we will be debugging and adding features merging with the
master branch upon reaching milestones such as movement, graphics, etc. Merging
conflicts with the central branch will be handled by the two members with
conflicts working together to ensure neither of their features are affected. In
short our workflow will consist of a central branch off of the master, with
conflicts resolved as a team.

\subsection*{Proof of Concept Demonstration plan:}  
The proof of concept
demonstration will show that we are capable of of completing what we set out to
do, we will prove this  by overcoming what we believe to be the most difficult
task. Since we plan to learn JavaScript throughout the development process we 
believe that the most difficult task will be to take user input through the
browser and into our game. The demo we plan to showcase will consist of a cube
which moves around the browser window, with respect to user input, with bounds
so that it does not go off screen. We are not deterred  with the task of
bringing our project to class in order to demonstrate for it is web-based and
runs in the browser window. For our proof of concept demo we will open an html
file which will run the script for our block with boundaries which will move
based on keyboard input.

\subsection*{Technology:} 
Technology has always been an important part of
software development. It is usually composed of four major parts which are the
programming language, the integrated development environment, the testing
framework and the document generation. For the part of programming language, the
plan is to build the structure of the website with HTML and CSS, running
JavaScript to complete the part of the game. As for IDE, we decided to use web-
storm and notepad++. The reason of choosing web-storm is because of its smart
and convenient features which are able to analyze the project to provide the
best code auto-completion results with hundreds of built-in inspections. And
notepad++ allows working with multiple open files in a single window which is
better than the built-in Windows text editor. Pertaining to the testing framework,
Qunit is a good choice for JavaScript which is similar to Junit which we have
experience with for Java. As for documentation, YUIdoc was chosen because it
can be used to document the code clearly and stably.

\subsection*{Coding Style:}  
Based on coding habits of all members in the
group, a unified coding style has been set up. We decided to use K\&R style for
the placement of the braces in compound statement. For the part of naming
variables, camelCase will be used throughout all the code which will make it
look uniform, constants will be written in all capital letters. Every class will
begin with a capital letter to distinguish them. Functions will end in 'Func' to
distinguish them. We also agreed to follow commenting standards to ensure that
when looking back at the code it will still be understandable. Indenting will be
mandatory and be one tab for levels of logic to be easily recognizable, for
example the code in a loop is indented to identify the code is running within
the loop. All global variables will be declared in the top of the document,
before any logic/code is implemented.

\subsection*{Project Schedule:} Refer to  
\href{https://gitlab.cas.mcmaster.ca/pathmap/JSTanks/tree/master/ProjectSchedule}
{Gantt chart here}

\subsection*{Project review (To be filled in: Revision 1):}

\end{document}