\documentclass{article}

\usepackage{booktabs}

\usepackage{fancyhdr}
\pagestyle{fancy}

\begin{document}

\lhead{\textbf{Team 6}:\\
	Jiahao, Li\\
	Pavithran, Pathmarajah\\
	Viren Patel }

\newpage

\title{Problem Statement}
\section*{\\\\Revision History}

\begin{table}[h!]
  \centering
  \caption{Revision History: Problem Statement}
  \label{tab:table1}
  \begin{tabular}{cccc}
	\toprule
	Date &  Developer & Change & Revision\\
	\midrule
	September 22&Jiahao Li &Initial Draft &0\\
	September 22&Pavithran Pathmarajah &Initial Draft &0\\
	September 22&Viren Patel  &Initial Draft &0\\
	December 7&Viren Patel &Final Draft&1\\
	\bottomrule
  \end{tabular}
\end{table}

\newpage

\section*{Problem Statement}

\subsection{What problem are you trying to solve?}
Video games are a means of entertainment used by all age groups, the trouble is for some of these groups the process of acquiring and runnning these games can be a hassle. We plan to contribute to solving this problem by reintroducing Tanks the game, but as an improved web based version. We are trying to make it more accessible to the entire user base as the accessibility is currently hindered by the need to download and compile the game on their computer if they wish to play. By transferring Tanks to the Web, we are making sure that the users are not restricted by the platform they are using. We are also eliminating the process of downloading and compiling, making it more convenient for all users. 

\subsection{Why is this an important problem?}
Traditionally it had been mandatory to optimize, compile and execute programs of even
minimal complexity for specific platforms. This was prior to the adoption of the World Wide Web,
this along with the exponential increase in computing power means people no longer need to
run programs designed for their specific systems. Today programs are made to run across
multiple platforms specifically Windows, Macintosh, and Linux. This was fine since it made up
most of the market. Today with the introduction computers such as the chromebook and
android based laptops, revive the original problem of platform dependent programs.

\subsection{What is the context of the problem you are solving?}
This game will draw interest from users of all ages on all platforms as our version of
Tanks will bring more features and have strong playability. Most devices today have access to a
web browser and also support JavaScript, by porting the game to the web we are essentially
converting from Java to Javascript and converting the game into a web structure which is built
on Html and Css. This also allows users to enjoy the game without downloading it, but at the
same time, the algorithm in the game will be modified to work fine on a web browser while
including more features within minimal effect to running time.

\end{document}